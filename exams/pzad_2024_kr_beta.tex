\documentclass[12pt]{article}

\usepackage{tikz} % картинки в tikz
\usepackage{microtype} % свешивание пунктуации
\usepackage{array} % для столбцов фиксированной ширины
\usepackage{comment} % для комментирования целых окружений
\usepackage{indentfirst} % отступ в первом параграфе

\usepackage{sectsty} % для центрирования названий частей
\allsectionsfont{\centering}

\usepackage{amsmath, amssymb, amsthm, amsfonts} % куча стандартных математических плюшек

\usepackage[top=2cm, left=1cm, right=1cm, bottom=2cm]{geometry} % размер текста на странице
\usepackage{lastpage} % чтобы узнать номер последней страницы
 
\usepackage{enumitem} % дополнительные плюшки для списков
%  например \begin{enumerate}[resume] позволяет продолжить нумерацию в новом списке

\usepackage{caption} % подписи к рисункам
\usepackage{hyperref} % гиперссылки
\usepackage{multicol} % текст в несколько столбцов


\usepackage{fancyhdr} % весёлые колонтитулы
\pagestyle{fancy}
\lhead{Прикладные задачи анализа данных, ВШЭ}
\chead{}
\rhead{13 июня 2024 г.}
\lfoot{Вариант $\beta$}
\rfoot{Паниковать запрещается!}
%\rfoot{Тест}
\renewcommand{\headrulewidth}{0.4pt}
\renewcommand{\footrulewidth}{0.4pt}

\usepackage{ifthen} % для написания условий

\usepackage{todonotes} % для вставки в документ заметок о том, что осталось сделать
% \todo{Здесь надо коэффициенты исправить}
% \missingfigure{Здесь будет Последний день Помпеи}
% \listoftodos --- печатает все поставленные \todo'шки


% более красивые таблицы
\usepackage{booktabs}
% заповеди из докупентации:
% 1. Не используйте вертикальные линни
% 2. Не используйте двойные линии
% 3. Единицы измерения - в шапку таблицы
% 4. Не сокращайте .1 вместо 0.1
% 5. Повторяющееся значение повторяйте, а не говорите "то же"


\usepackage{fontspec}
\usepackage{polyglossia}

\setmainlanguage{russian}
\setotherlanguages{english}

% download "Linux Libertine" fonts:
% http://www.linuxlibertine.org/index.php?id=91&L=1
\setmainfont{Linux Libertine O} % or Helvetica, Arial, Cambria
% why do we need \newfontfamily:
% http://tex.stackexchange.com/questions/91507/
\newfontfamily{\cyrillicfonttt}{Linux Libertine O}

% Математические шрифты 
% Математические шрифты 
\usepackage{unicode-math}     
\setmathfont[math-style=upright]{euler.otf} 

\setmathfont[range={\mathbb, \mathop, \heartsuit, \angle, \smile, \varheartsuit}]{Asana-Math.otf}

\AddEnumerateCounter{\asbuk}{\russian@alph}{щ} % для списков с русскими буквами
\setlist[enumerate, 2]{label=\asbuk*),ref=\asbuk*}


% мои цвета https://www.artlebedev.ru/colors/
\definecolor{titleblue}{rgb}{0.2,0.4,0.6} 
\definecolor{blue}{rgb}{0.2,0.4,0.6} 
\definecolor{red}{rgb}{1,0,0.2} 
\definecolor{green}{rgb}{0,0.6,0} 
\definecolor{purp}{rgb}{0.4,0,0.8} 

% цвета из geogebra 
\definecolor{litebrown}{rgb}{0.6,0.2,0}
\definecolor{darkbrown}{rgb}{0.75,0.75,0.75}

% Гиперссылки
\usepackage{xcolor}   % разные цвета

\usepackage{hyperref}
\hypersetup{
    unicode=true,           % позволяет использовать юникодные символы
    colorlinks=true,        % true - цветные ссылки
    urlcolor=blue,          % цвет ссылки на url
    linkcolor=black,          % внутренние ссылки
    citecolor=green,        % на библиографию
    breaklinks              % если ссылка не умещается в одну строку, разбивать её на две части?
}

% эпиграфы
\usepackage{epigraph}
\setlength\epigraphwidth{.6\textwidth}
\setlength\epigraphrule{0pt}

% Математические операторы первой необходимости:
\DeclareMathOperator{\sgn}{sign}
\DeclareMathOperator*{\argmin}{arg\,min}
\DeclareMathOperator*{\argmax}{arg\,max}
\DeclareMathOperator{\Cov}{Cov}
\DeclareMathOperator{\Var}{Var}
\DeclareMathOperator{\Corr}{Corr}
\DeclareMathOperator{\E}{\mathop{E}}
\DeclareMathOperator{\Med}{Med}
\DeclareMathOperator{\Mod}{Mod}
\DeclareMathOperator*{\plim}{plim}

\DeclareMathOperator{\logloss}{logloss}
\DeclareMathOperator{\softmax}{softmax}

\DeclareMathOperator{\tr}{tr}

% команды пореже
\newcommand{\const}{\mathrm{const}}  % const прямым начертанием
\newcommand{\iid}{\sim i.\,i.\,d.}  % ну вы поняли...
\newcommand{\fr}[2]{\ensuremath{^{#1}/_{#2}}}   % особая дробь
\newcommand{\ind}[1]{\mathbbm{1}_{\{#1\}}} % Индикатор события
\newcommand{\dx}[1]{\,\mathrm{d}#1} % для интеграла: маленький отступ и прямая d

% одеваем шапки на частые штуки
\def \hb{\hat{\beta}}
\def \hs{\hat{s}}
\def \hy{\hat{y}}
\def \hY{\hat{Y}}
\def \he{\hat{\varepsilon}}
\def \hVar{\widehat{\Var}}
\def \hCorr{\widehat{\Corr}}
\def \hCov{\widehat{\Cov}}

% Греческие буквы
\def \a{\alpha}
\def \b{\beta}
\def \t{\tau}
\def \dt{\delta}
\def \e{\varepsilon}
\def \ga{\gamma}
\def \kp{\varkappa}
\def \la{\lambda}
\def \sg{\sigma}
\def \tt{\theta}
\def \Dt{\Delta}
\def \La{\Lambda}
\def \Sg{\Sigma}
\def \Tt{\Theta}
\def \Om{\Omega}
\def \om{\omega}

% Готика
\def \mA{\mathcal{A}}
\def \mB{\mathcal{B}}
\def \mC{\mathcal{C}}
\def \mE{\mathcal{E}}
\def \mF{\mathcal{F}}
\def \mH{\mathcal{H}}
\def \mL{\mathcal{L}}
\def \mN{\mathcal{N}}
\def \mU{\mathcal{U}}
\def \mV{\mathcal{V}}
\def \mW{\mathcal{W}}

% Жирные буквы
\def \mbb{\mathbb}
\def \RR{\mbb R}
\def \NN{\mbb N}
\def \ZZ{\mbb Z}
\def \PP{\mbb{P}}
\def \QQ{\mbb Q}

\def \putyourname{\fbox{
    \begin{minipage}{42em}
      Фамилия, имя, номер группы:\vspace*{3ex}\par
      \noindent\dotfill\vspace{2mm}
    \end{minipage}
  }
}

\def \checktable{

    \vspace{5pt}
    Табличка для проверяющих работу:

\vspace{5pt}

    \begin{tabular}{|m{2cm}|m{1cm}|m{1cm}|m{1cm}|m{1cm}|m{1cm}|m{1cm}|m{1cm}|m{2cm}|}
\toprule
        Задачи & 1 & 2 & 3 & 4 & 5 & 6 & 7 & Итого \\
\midrule
        &  &  & & & & & & \\
        &  &  & & & & & & \\
 \bottomrule
\end{tabular}
}


\def \testtable{

\vspace{5pt}
    Внесите сюда ответы на тест:

\vspace{5pt}

\begin{tabular}{|m{2cm}|m{0.6cm}|m{0.6cm}|m{0.6cm}|m{0.6cm}|m{0.6cm}|m{0.6cm}|m{0.6cm}|m{0.6cm}|m{0.6cm}|m{0.6cm}|}
\toprule
        Вопрос & 1 &  2 & 3 & 4 & 5 & 6 & 7 & 8 & 9 & 10 \\
\midrule
        Ответ &  &  & & & & & & & & \\
 \bottomrule
\end{tabular}
}


% [1][3] 1 = one argument, 3 = value if missing
% эта магия создаёт окружение answerlist
% именно в окружении answerlist записаны варианты ответов в подключаемых exerciseXX
% просто \begin{answerlist} сделает ответы в три столбца
% если ответы длинные, то надо в них руками сделать
% \begin{answerlist}[1] чтобы они шли в один столбец
\newenvironment{answerlist}[1][3]{
\begin{multicols}{#1}

\begin{enumerate}[label=\fbox{\emph{\Alph*}},ref=\emph{\alph*}]
}
{
\item Нет верного ответа.
\end{enumerate}
\end{multicols}
}

% BB: unicol version. don't know why \ifthenelse fails in second part of new-env
\newenvironment{answerlistu}{
\begin{enumerate}[label=\fbox{\emph{\Alph*}},ref=\emph{\alph*}]
}
{
\item Нет верного ответа.
\end{enumerate}
}


\excludecomment{solution} % without solutions

\theoremstyle{definition}
\newtheorem{question}{Вопрос}

\usepackage{tikzlings}
\usepackage{tikzducks}

\usepackage{alltt}

\begin{document}

\putyourname

% \testtable

% \checktable

\epigraph{Мы выживем назло Вселенной. Нас не сломить.}{\textit{(Элли из Last of us перед тем как решить эту контрольную)}}

Работа состоит из открытых вопросов на разные темы. На каждый из них вам необходимо дать краткие, но ёмкие ответы. Около каждого вопроса указано количество баллов, которое можно за него получить. Если у вопроса несколько подпунктов, баллы разделяются между  ними равномерно.

\begin{question} \textbf{(2.5 балла)} \newline
    Начался зомби-апокалипсис. Гриб кордицепс мутировал из-за глобального потепления и полностью подчиняет себе нервную систему заражённых людей.

    Билл урылся от зомби у себя дома. Он превратил его в крепость и \textbf{живёт один.} Ему скучно. Мимо проходил выпускник ИАДа, то есть вы. Билл достал свою двустволку и вежливо попросил вас обучить ему рекомендательную систему для фильмов, чтобы можно было коротать время.
    \begin{enumerate}
        \item Получится для одного Билла обучить user-based колаборативную фильтрацию? А content-based? А матричную факторизацию? 
        
        \item Что такое контентные рекомендации? Как можно обучить такую рекомендательную систему для Билла? 

        \item Что такое проблема холодного старта? Будет ли она актуальна для рекомендательной системы Билла? Как её решить?  
        
        \item Что такое feedback loop? Будет ли эта проблема актуальна для рекомендательной системы Билла? Как её решить? 
    \end{enumerate}
\end{question}

\begin{question} \textbf{(2.5 балла)} \newline
    Начался зомби-апокалипсис. Гриб кордицепс мутировал из-за глобального потепления и полностью подчиняет себе нервную систему заражённых людей.

    Люди концентрируются в городах и огораживают их огромными стенами. В городах процветает чёрный рынок. В том числе чёрный рынок нейросетей. Вас, выпускника ИАДа, наняли как эксперта, чтобы разобраться с проблемами:

    \begin{enumerate}
        \item У торговцев есть GAN, обученный на датасете со многими видами животных, но он генериует только картинки котиков. Какая проблема произошла при обучении? Проиллюстрируйте ее для двумерного случая.

        \item Один из торговцев нашел на флэшке дискриминатор из другого GAN. Он предлагает после обычного обучения «дообучить» ваш GAN с его помощью. Как это может помочь и в чём опасность такого подхода? 
        
        \item Чёрные дельцы предлагают для нормализационного потока использовать нейронную сеть, которая принимает на вход картинки $256 \times 256$ и возвращает вектор размерности $64$. Почему такую модель не получится обучить?
    \end{enumerate}
\end{question}

\newpage

\begin{question} \textbf{(2.5 балла)} \newline
    Начался зомби-апокалипсис. Гриб кордицепс мутировал из-за глобального потепления и полностью подчиняет себе нервную систему заражённых людей.

    Чтобы лучше взаимодействовать между собой, зомби создали социальную сеть. Они поймали выпускинка ИАДа, то есть вас и просят помочь сделать её лучше. Пока вы им помогаете, вас обещали не есть.

    Зомби хотят реализовать внутри социальной сети механизм рекомендации нового друга. В этом им могут помочь графы. Каждая вершина -- это пользователь социальной сети. Ребро между ними означает, что пользователи добавили друг-друга в друзья.

    \begin{enumerate}
        \item Опишите алгоритм Node2Vec для обучения представлений вершин графа. 
        
        \item Предположим, что мы используем в качестве функции похожести вероятность перехода из одной вершины в другую при случайном блуждании по графу. Опишите как вы будете готовить выборку для обучения Node2Vec, используя случайное блуждание. 
        
        \item Как при обучении представлений вершин графа можно учесть дополнительные факторы? Например, информацию со странички пользователя, его пол возраст и тп. 
        
        \item Вы обучили для каждой вершины графа эмбеддинги. Теперь вам нужно обучить модель, которая по этим эмбеддингам будет предсказывать, что два конкретных человека добавят друг-друга в друзья. Как собрать данные для обучения такой модели? Что будет таргетом? Что будет факторами? Какую модель вы обучите и как будете строить предсказание? 
    \end{enumerate}
\end{question}

\begin{question} \textbf{(2.5 балла)} \newline
    Начался зомби-апокалипсис. Гриб кордицепс мутировал из-за глобального потепления и полностью подчиняет себе нервную систему заражённых людей.

    Лучшие умы планеты (выпускники ИАДа) расшифровали язык зомби и пытаются обучить speech2text для него. Вы руководитель исследовательской группы.     
    \begin{enumerate}
        \item  Опишите пайплайн, для решения такой задачи. Как вы предобработаете аудиозаписи перед обучением модели? Какую архитектуру будете использовать? Как будете разбивать данные на обучающие и тестовые?
        
        \item Какие метрики качества для задачи распознавания текста в аудио вы знаете? Как их можно вычислить на тестовой выборке?
        
        \item Вы обучили модель и после этого выяснили, что все данные для обучения собирались в идеальных лабораторных условиях. В полевых условиях микрофон улавливает кучу шума и распознавание речи работает на порядок хуже. Бюджета на сбор новых данных нет. Как бы вы решили эту проблему, используя старые данные?
    \end{enumerate}
\end{question}

\end{document}

